\documentclass[11pt]{article}
\usepackage[margin=2.5cm]{geometry}
\usepackage{graphicx}
\usepackage{amsthm, amsmath, amssymb}
\newtheorem{theorem}{Theorem}
\theoremstyle{definition}
\newtheorem{definition}{Definition}
\theoremstyle{remark}
\newtheorem{example}{Example}


%For urls
\usepackage{url}

\title{Model Checking Real-Time Systems\\\small Written Abstract for the Seminar ``Recent Advances in Model Checking''}
\author{Vincent Trélat}
\date{}

\begin{document}
\maketitle	

\section*{Organizational information}
% - organization of the abstract and relation to the talk
% - motivation of why the others should care
% - formalities that you plan to use in your talk
% - an outlook of what to expect in your talk 
% - if possible, a paragraph on the relation to (some of) the other talks

\section{Introduction}\label{sec:intro}

\section{Preliminaries}\label{sec:prelims}
In this chapter, time values are equated with non-negative real numbers of $\mathbb{R}_{\geq 0}$. A \emph{time sequence} is a finite or infinite non-decreasing sequence of time values. A \emph{timed word} over some alphabet $\Sigma$ is a finite or infinite sequence of pairs of $\Sigma \times \mathbb{R}_{\geq 0}$ such that the sequence formed with the second components of each pair is a time sequence. If the time sequence of a timed word is upper-bounded or converging, the timed word is said to be \emph{converging}.

Let $C$ be a finite set of variables called \emph{clocks}. A \emph{valuation} over $C$ is a mapping $v \colon C \to \mathbb{R}_{\geq 0}$. The set of valuations over $C$ is denoted by $\mathbb{R}_{\geq 0}^C$ and $\text{\bf 0}_C$ denoted the valuation assigning 0 to every clock of C.

For any valuation $v$ and any time value $t$, the valuation $v + t$ denotes the valuation obtained by shifting all values of $v$ by $t$. For any subset $r$ of $C$, $v[r]$ is the valuation obtained by resetting all clocks of $r$ in $v$ (i.e.\ set them to 0).

A \emph{constraint} $\varphi$ over $C$ is recursively defined as follows:
\begin{itemize}
\item if $x \in C$, $k \in \mathbb{Z}$ and $\odot \in \{<, \leq, =, \geq, >\}$, then $x \odot k$ is a constraint over $C$,
\item if $\varphi_1$ and $\varphi_2$ are constraints over $C$, then $\varphi_1 \land \varphi_2$ is a constraint over $C$.
\end{itemize}
The set of constraints over $C$ is denoted by $\Phi(C)$.
We say that a valuation $v$ over $C$ satisfies $x \odot k$ when $v(x) \odot k$, and when $v$ satisfies a constraint $\varphi$, we write $v \models \varphi$. The set of valuations satisfying a constraint $\varphi$ is denoted by $[\![\varphi]\!]_C$.

\section{Timed Automata}\label{sec:ta}
\begin{definition}\label{def:ta}
	A \emph{Timed Automaton} (TA) is a tuple $\mathcal{A} = (L, l_0, C, \Sigma, I, E)$ where:
	\begin{itemize}
		\item $L$ is a finite set of \emph{locations} with initial location $l_0 \in L$;
		\item $C$ is a finite set of \emph{clocks};
		\item $\Sigma$ is a finite set of \emph{actions};
		\item $I \colon L \to \Phi(C)$ is an \emph{invariant mapping};
		\item $E \subseteq L \times \Phi(C) \times \Sigma \times 2^{C} \times L$ is a set of edges.
	\end{itemize}
	Any edge $(\ell, \varphi, a, r, \ell') \in E$ is denoted by $\ell \xrightarrow{\varphi, a, r} \ell'$ where $\varphi$ is a \emph{guard}, and $r$ is a subset of clocks that are set to zero after taking the transition.
\end{definition}

\bibliographystyle{alpha}
\bibliography{ref}

\end{document}
